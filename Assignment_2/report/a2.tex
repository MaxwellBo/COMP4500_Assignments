\documentclass[11pt,a4paper]{article}

\usepackage{fullpage}
\usepackage[margin=1in]{geometry}
\usepackage[utf8]{inputenc}
\usepackage{amsmath}
\usepackage{amssymb}
\usepackage{verbatim}

\usepackage{graphicx}
\graphicspath{ {images/} }

\usepackage[colorlinks = true,
            linkcolor = blue,
            urlcolor  = blue,
            citecolor = blue,
            anchorcolor = blue]{hyperref}

\begin{document}

\title{COMP4500 \\ Assignment 2}
\author{Maxwell Bo (43926871)}
\date{October 19, 2018}
\maketitle

\subsection*{Question 1}
\subsubsection*{(a)}

\verbatiminput{1a.py}

\subsubsection*{a.}

Consider some arbitrary list of track requests. 

Assume $H \leqslant \min(requests)$. The optimal ordering is such that $requests$ is sorted in ascending order, moving to $\min(requests)$, sweeping across all the intermediate tracks until it lands at $\max(requests)$. The head had to move $(\max(requests) - \min(requests)) + (\min(requests) - H)$ tracks.

Assume $\max(requests) \leqslant H$. The optimal ordering is such that $requests$ is sorted in descending order, moving to $\max(requests)$, sweeping across all the intermediate tracks until it lands at $\min(requests)$. The head had to move $(\max(requests) - \min(requests)) + (H - \max(requests))$ tracks.

Let the $midpoint(requests) = (\min(requests) + \max(requests)) / 2$. We want to minimize the distance we move to the start or end of the sorted $requests$ before we start our sweep.

Assume $\min(requests) \leqslant H \leqslant midpoint$. The optimal ordering is such that $requests$ is sorted in ascending order, moving to $\min(requests)$, sweeping across all the intermediate tracks until it lands at $\max(requests)$. The head had to move $(\max(requests) - \min(requests)) + (H - \min(requests))$ tracks.

Assume $midpoint \leqslant H \leqslant \max(requests)$. The optimal ordering is such that $requests$ is sorted in descending order, moving to $\max(requests)$, sweeping across all the intermediate tracks until it lands at $\min(requests)$. The head had to move $(\max(requests) - \min(requests)) + (\max(requests)- H)$ tracks.

Thus, our optimal cost is always the range of the requests, plus the the distance travelled to get to either the minimum or the maximum of the requests, whichevers smallest.

\newpage

\subsection*{Question 2}

\subsubsection*{b.}

The recursive procedure forms a tree with branching factor 3, with each branch created by either \textit{fully rebooting}, \textit{partially rebooting} or opting not to reboot. 

At the root, we have 1 node. On the next level, we have 3, and on the next level, we have 9. This is the geometric series.

To calculate the total number of nodes in the tree after $k$ depth, we sum the geometric series.

\begin{align*}
            \sum_{i = 0}^{k} 3^i
        &= 1 + 3 + 3^2 + \ldots + 3^k\\
        & = \dfrac{3^{k + 1} - 1}{3 - 1}\\
        & = \dfrac{3^{k + 1} - 1}{2}
\end{align*}

This is $O(3^k)$.

\subsubsection*{d.}

By peeking at the implementation of the bottom-up table construction procedure, we can see:

\begin{verbatim}
for (int d = lastDay; d >= 0; d--) {
    for (int i = 0; i <= d + 1; i++) {
        // add to table
    }
}
\end{verbatim}

which is equivalent to

\begin{align*}
            \sum_{d = 0}^{k - 1} \sum_{i = 0}^{d} 1
            & = \sum_{d = 1}^{k} \sum_{i = 1}^{d + 1} 1\\
            & = \sum_{d = 1}^{k} (d + 1)\\
            & = \sum_{d = 1}^{k} d + \sum_{d = 1}^{k} 1\\
            & = (\sum_{d = 1}^{k} d) + k\\
            & = \dfrac{k(k + 1)}{2} + k\\
            & = \dfrac{k^2 + k}{2} + k\\
            & = \dfrac{k^2 + 3k}{2}
\end{align*}

Therefore, we can see the upper-bound is $\Omega(k^2)$.

\end{document}
